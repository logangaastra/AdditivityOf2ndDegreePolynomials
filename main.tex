\documentclass{article}
\usepackage[utf8]{inputenc}
\usepackage[english]{babel}
\usepackage{amssymb}
\usepackage{amsmath}
\usepackage{amsthm}

\title{Additivity of 2nd degree polynomials}
\author{Logan Gaastra }
\date{February 2017}

\newtheorem*{definition}{Definition}
\newtheorem{theorem}{Theorem}[section]
\newtheorem{lemma}{Lemma}

\begin{document}

\maketitle

\section{Introduction}
A Pythagorean triple is a triplet of integers (x,y,z) such that $ x^2+y^2=z^2 $. The idea of a Pythagorean triple was created by Pythagoras through his study of right triangles and fascination with integers, with a Pythagorean triple representing a right triangle with sides of integer lengths.

This paper attempts to purport not only the existence of triplets for the polynomial $x^2$ but also for all 2nd degree polynomials.
\section{Additive Triplets}
\begin{definition}{Additive Triplets:}
Let $AP(a_0,a_1,...,a_n)$ denote the set of integer triplets (x,y,z) such that for nth degree polynomial 
$P(x) = a_0 + a_1x+...+a_nx^n$, $$P(x) + P(y) = P(z)$$
We will call this the set of additive triplets for a polynomial.
\end{definition}
 We note that if $(x,y,z) \in  AP(a_0,..,a_n) $, then $(y,x,z) \in AP(a_0,..,a_n) $, which fairly obviously follows from the additivity of integers. We will be focusing specifically on the case for 2nd degree polynomials, which we will denote as AP(a,b,c).

Let $(x,y,z) \in  AP(a,b,c) $, then $$ax^2+bx^2+c+ay^2+by+c=az^2+bz+c$$ Further, $$a(z^2-x^2-y^2)+b(z-x-y)=c$$ So if we can show that there exist integers x,y,z that satisfy the previous equation, then we can show that AP(a,b,c) is not empty.

\begin{lemma}
if and only if $i \equiv j(mod  2)$, there exists integers x,y,z such that $x^2+y^2-z^2 = i$, $x+y-z=j$
\end{lemma}

Pf: We begin by noting that $x^2+y^2-z^2\equiv x+y-z(2)$ so then for $i\equiv j+1(2)$ there will not exist x,y,z to satisfy the equalities.

Let i and j be odd, then i = 2i' + 1, j = 2j' + 1 for some i', j'. Then
$$(2j')^2+(i'-2j'^2+1)^2-(i'-2j'^2)^2= 2i'+1 = i$$ $$(2j')+(i'-2j'^2+1)-(i'-2j'^2) = 2j'+1 = j$$
So then $(x,y,z)=(2j',i'-2j'^2+1,i'-2j'^2)$ solves the equalities.

Now let i and j be even, then i = 2i', j = 2j' for some i', j'. Then 
$$(2j'-1)^2+(2j'^2+3j'-i')^2-(2j'^2-3j'-i'-1)^2 = 2i'=i$$
$$(2j'-1)+(2j'^2+3j'-i')-(2j'^2-3j'-i'-1) = 2j'=j$$
So then $(x,y,z) = (2j',2j'^2-3j'-i',2j'^2-3j'-i'-1)^2$ solves the inequalities. So then for both even and odd i, j there exists integer x,y,z that solves the equalities $x^2+y^2-z^2 = i$, $x+y-z=j$

\begin{lemma} %Lemma 2 
\emph{Bezouts Identity: }
For the equation ai+bj=c, if gcd(a,b) does not divide c, then the equation has no solutions.
\end{lemma}

\begin{lemma} %Lemma 3
If ai+bj=1 has solutions, exclusively either there exist integers i', j' such that $i'\equiv j'\text{(mod  2)}$ or $a\equiv b\text{(mod  2)}$
\end{lemma}
If ai+bj=1 has solutions, then gcd(a,b)=gcd(i,j)=1, so then $i\not\equiv j \text{(mod  2)}$ or $a\not\equiv b\text{(mod  2)}$. 

If only $a\not\equiv b\text{(mod  2)}$ then we need not prove any further.

If $a\not\equiv b \text{(mod  2)}$ and $i\not\equiv j \text{(mod  2)}$ then $i\equiv a\text{(mod  2)}$ and $j\equiv b\text{(mod  2)}$. Further, $i\not\equiv j \equiv b\text{(mod  2)}$ and also $j\not\equiv i \equiv a \text{(mod  2)}$. Then a(i+b)+b(j-a)=1 is also true, and $i+b\equiv j-a\text{(mod  2)}$.

Finally, if $a\equiv b\text{(mod  2)}$, then i, j must not both be odd or even, or else ai+bj would be even. 

\begin{theorem}
For $P(x)=ax^2+bx+c$ there exist x,y,z such that P(x)+P(y)=P(z) if and only if $gcd(a,b)\mid c$ and either $2*gcd(a,b)\mid c$ or $\frac{ab}{gcd(a,b)^2}$ is even. This is equivalent to saying that P(x) +P(y)+P(z) if and only if $\frac{abc}{2gcd(a,b)^3}\in \mathbb{Z}$
\end{theorem}
P(x)+P(y)=P(z) for $P(x)=ax^2+bx+c$ is equivalent to $a(z^2-x^2-y^2)+b(z-x-y)=c$. Using Lemma 2, we know that gcd(a,b)|c must be true for the equation to have solutions, so we need not prove that any further.

Continuing, let $a'=\frac{a}{gcd(a,b)}$ and $b'=\frac{b}{gcd(a,b)}$, $c'=\frac{c}{gcd(a,b)}$ then gcd(a',b')=1, so then there exists i,j such that a'i+b'j=1, further a'(ic')+b'(jc')=c'. So now, we just need to show that there exists x,y,z  such that $ic' = z^2-x^2-y^2$ and $jc'=z-x-y$.

Let c' be divisible by 2, then ic' and jc' are equivalent modulo 2, so then using Lemma 1, there must exist solutions to $ic' = z^2-x^2-y^2$ and $jc'=z-x-y$.

Now let $2|\frac{ab}{gcd(a,b)^2}$, then 2|a'b', so then $a'\equiv b' + 1(mod 2)$. By Lemma 3, since $a\equiv b+1(mod2)$, there must exist i',j' such that $i'\equiv j'(mod2)$. As such, $i'c'\equiv j'c'(mod2)$ so then by Lemma 1 there must exist solutions such that $ic' = z^2-x^2-y^2$ and $jc'=z-x-y$.

For the case in which neither c' nor $\frac{ab}{gcd(a,b)^2}$ is divisible by 2, a corollary of $\frac{ab}{gcd(a,b)^2}$ not being divisible by 2 is that $a\equiv b(mod 2)$ and by Lemma 3, since $a\equiv b(mod 2)$, there does not exist i',j' such that $i'\equiv j'(mod 2)$. Also, c' is odd, so then $ic'\equiv i(mod 2)$ and $jc'\equiv j(mod 2)$ so then $ic'\equiv i\equiv j + 1 \equiv jc' + 1(mod 2)$ and by Lemma 1, since ic' is not equivalent to jc' modulus 2, there do not exist solutions to $x^2+y^2-z^2 = ic'$, $x+y-z=jc'$.
$$QED$$


\subsection{Additional Theorem}
\begin{theorem}
If (x,y,z) is contained in AP(a,b,c) for any a,b,c, then $(z-x)\mid P(y)$ and $(z-y) \mid P(x)$
\end{theorem}
We know that if P(x)+P(y)=P(z) then $a(z^2-x^2-y^2)+b(z-x-y)=c$. so then
$$a(z^2-x^2-y^2)+b(z-x-y)=c$$
$$a(z^2-y^2)+b(z-y)= P(x)$$
$$P(x) =a(z-y)(z+y)+b(z-y)$$
$$P(x) =(z-y)a((y+z)+b)$$
$$(z-y)\mid P(x)$$
You can use the same technique to arrive at the other result.
$$QED$$
\end{document}
